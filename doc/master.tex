\documentclass[a4paper, 11pt, toc=bibliography, toc=listof]{scrbook}

% Preamble {{{
\usepackage[ngerman]{babel}
\usepackage[utf8]{inputenc}
\usepackage[T1]{fontenc}
\usepackage[pdftex]{graphicx}
\usepackage[%
	colorlinks=false,
	pdfborder={0 0 0},
]{hyperref}
\usepackage{epstopdf}
\usepackage{bibgerm}
\usepackage{setspace}
\usepackage{amsmath, amsthm, amssymb}
\usepackage{scrpage2}

% Titlepage {{{
\usepackage[absolute]{textpos}
\usepackage{color}

\newlength{\TitleMargin}
\newlength{\TitleWidth}

\setlength{\TitleMargin}{2cm}
\setlength{\TitleWidth}{\paperwidth}
\addtolength{\TitleWidth}{-\TitleMargin}
\addtolength{\TitleWidth}{-\TitleMargin}

\definecolor{uniblue}{rgb}{0.062745,0.17647,0.34118}

\newcommand{\TitleUni}{Universität Potsdam}
\newcommand{\TitleInstitut}{Mathematisch-Naturwissenschaftliche Fakultät\\Institut für Informatik}
\newcommand{\TitleTitel}{Selbst-adaptive Lastverteilung für DNS-Clusters}
\newcommand{\TitleTyp}{Masterarbeit}
\newcommand{\TitleAutor}{Sebastian Menski}
\newcommand{\TitleBetreuerText}{Betreuer}
\newcommand{\TitleBetreuer}{Prof. Dr. Bettina Schnor\\ &M.Sc. Jörg Zinke}
\newcommand{\TitleAbschlussText}{zur Erlangung des akademischen Grades\\Master of Science\\in Informatik}
\newcommand{\TitleOrt}{Potsdam}
\newcommand{\TitleDatum}{21. Juni 2012}
\hypersetup
{
	pdfauthor={Sebastian Menski},
	pdftitle={\TitleTitel},
}

\renewcommand{\maketitle}{
	\thispagestyle{empty}
	\begin{textblock*}{\TitleWidth}(\TitleMargin,\TitleMargin)
		~\hfill\includegraphics[height=2.5cm]{images/uni-logo}\\[3mm]
		{\color{uniblue}\rule{\TitleWidth}{1mm}}\\[5mm]
		{
			\centering
			\sffamily\Large
			{\LARGE\TitleUni}\\[0.5\baselineskip]
			{\large\TitleInstitut}\\[5\baselineskip]
			{\Huge\TitleTitel}\\[3\baselineskip]

			{\TitleTyp}\\
			\TitleAbschlussText\\[3\baselineskip]

			\TitleAutor\\[6\baselineskip]
			\begin{tabular}{rl}
				\TitleBetreuerText: & \TitleBetreuer
			\end{tabular}\\[3\baselineskip]
			\TitleOrt, \TitleDatum\par
		}
	\end{textblock*}
	~\clearpage
}

% Titlepage }}}

% Preamble }}}


\begin{document}
	 % Frontmatter {{{
	\frontmatter
	\maketitle{}
	\tableofcontents{}
	% Frontmatter }}}

	% Mainmatter {{{
	\onehalfspacing{}
	\mainmatter
	\pagestyle{scrheadings}

	\chapter{Einleitung} % {{{
	\label{cha:Einleitung}

		\section{Motivation} % {{{
		\label{sec:Motivation}
			
		% section Motivation }}}

		\section{Aufgabenstellung} % {{{
		\label{sec:Aufgabenstellung}
			
		% section Aufgabenstellung }}}

		\section{Gliederung der Arbeit} % {{{
		\label{sec:Gliederung der Arbeit}
			
		% section Gliederung der Arbeit }}}

	% chapter Einleitung }}}

	\chapter{Grundlagen} % {{{
	\label{cha:Grundlagen}

		\section{Domain Name System (DNS)} % {{{
		\label{sec:Domain Name System (DNS)}
			
		% section Domain Name System (DNS) }}}

		\section{Lastverteilung} % {{{
		\label{sec:Lastverteilung}
			
		% section Lastverteilung }}}

		\section{salbnet} % {{{
		\label{sec:salbnet}
			
			\subsection{salbd} % {{{
			\label{sub:salbd}
				
			% subsection salbd }}}

			\subsection{servload} % {{{
			\label{sub:servload}
				
			% subsection servload }}}

		% section salbnet }}}

  % chapter Grundlagen }}}

	\chapter{Verwandte Arbeiten} % {{{
	\label{cha:Verwandte Arbeiten}
		
	% chapter Verwandte Arbeiten }}}

	\chapter{Implementation} % {{{
	\label{cha:Implementation}
		
		\section{Anforderungen} % {{{
		\label{sec:Anforderungen}
			
		% section Anforderungen }}}

		\section{Erweiterung von servload} % {{{
		\label{sec:Erweiterung von servload}
			
		% section Erweiterung von servload }}}

		\section{Erweiterung von salbd} % {{{
		\label{sec:Erweiterung von salbd}
			
		% section Erweiterung von salbd }}}

	% chapter Implementation }}}

	\chapter{Funktionstest} % {{{
	\label{cha:Funktionstest}
	
		\section{Testumgebung} % {{{
		\label{sec:Testumgebung}
			
		% section Testumgebung }}}

		\section{Messplan} % {{{
		\label{sec:Messplan}
			
		% section Messplan }}}

		\section{Auswerterung} % {{{
		\label{sec:Auswerterung}
			
		% section Auswerterung }}}

	% chapter Funktionstest }}}

	\chapter{Zusammenfassung} % {{{
	\label{cha:Zusammenfassung}
	
	% chapter Zusammenfassung }}}

	% Content }}}

	% Appendix {{{
	\appendix

	\chapter{Messumgebung} % {{{
	\label{cha:Messumgebung}

	% chapter Messumgebung }}}

	% Appendix }}}

	% Backmatter {{{
	\backmatter
	\pagenumbering{Roman}

	\listoffigures{}
	\listoftables{}

	% Bibliography {{{
	\nocite{*}
	\bibliographystyle{gerplain}
	\bibliography{references}
	% }}}

	% Backmatter }}}

\end{document}
